\documentclass{article}

% if you need to pass options to natbib, use, e.g.:
% \PassOptionsToPackage{numbers, compress}{natbib}
% before loading nips_2016
%
% to avoid loading the natbib package, add option nonatbib:
% \usepackage[nonatbib]{nips_2016}

% \usepackage{nips_2016}

% to compile a camera-ready version, add the [final] option, e.g.:
\usepackage[final]{nips_2016}

\usepackage[utf8]{inputenc} % allow utf-8 input
\usepackage[T1]{fontenc}    % use 8-bit T1 fonts
\usepackage{hyperref}       % hyperlinks
\usepackage{url}            % simple URL typesetting
\usepackage{booktabs}       % professional-quality tables
\usepackage{amsfonts}       % blackboard math symbols
\usepackage{nicefrac}       % compact symbols for 1/2, etc.
\usepackage{microtype}      % microtypography

\usepackage{enumitem}
\usepackage{xfrac}

\title{Your Awesome Project Title}

% The \author macro works with any number of authors. There are two
% commands used to separate the names and addresses of multiple
% authors: \And and \AND.
%
% Using \And between authors leaves it to LaTeX to determine where to
% break the lines. Using \AND forces a line break at that point. So,
% if LaTeX puts 3 of 4 authors names on the first line, and the last
% on the second line, try using \AND instead of \And before the third
% author name.

\author{
  Name 1\\
  Department 1\\
  \texttt{id1@andrew.cmu.edu} \\
  \And
  Name 2 \\
  Department 2\\
  \texttt{id2@andrew.cmu.edu} \\
  \And
  Name 3\\
  Department 3\\
  \texttt{id3@andrew.cmu.edu} \\
}

\begin{document}

\maketitle

% \begin{abstract}
% No abstract.
% \end{abstract}
Your final report should be 8 pages long (excluding references) and include the following:

\section*{Instructions}



\begin{itemize}[leftmargin=2em]
    

    \item \textbf{Introduction (1--2 pages):}
    Problem statement and literature overview; how your idea fits into the existing work.
    If you already had an elaborate proposal/midway report, you may borrow this part from there, but try to make it focused and concise.
    \emph{If you changed the topic of your project, make sure your introduction matches the problem you study.}

\end{itemize}

\section*{Methods}

\begin{itemize}[leftmargin=2em]
 \item \textbf{Methods (2--4 pages):}
    If you are developing a new method / model / architecture, this section should contain all the math, derivations, formal details, illustrations, etc.
    If you are proving theorems, assumptions, theorem formulations and proofs should be here.
    \emph{Even if your project is purely application driven, methods should include enough details about the models you are using to make your reports self-contained.}	
\end{itemize}




\section*{Results}
\begin{itemize}[leftmargin=2em]
 \item \textbf{Results (2--4 pages):}
    Your experimental results and discussion of thereof.
    Describe how the current results in each of the experiments align with your expectations.
    You may include the details on how you mined / prepared the data, if it is an important aspect of the project.
\end{itemize}

\section*{Discussion}

\begin{itemize}[leftmargin=2em]
	


\item \textbf{Discussion (\sfrac{1}{2}--1 page):}
    Please discuss the results of your project, including:
    \begin{itemize}
        \item Reiterate the main achievements of your project (proved theorems, interesting results, insights about the problem you were working on, etc.).
        \item Highlight a few limitations of your approach (e.g., strong assumptions you had to make, constraints, when your method didn't work in practice, etc.). Comment on whether you think there is a way to further improve your method to eliminate these limitations.
    \end{itemize}
\end{itemize}
    
\section*{Additional notes}

Please make sure you address the comments on your midway reports (especially if they specific actions/details are requested).

\section*{Grading rubric}

\textbf{Total weight:} 75\% of the project grade (i.e., 30\% of the course grade).

\begin{itemize}[leftmargin=2em]
    \item Introduction \& literature survey (10\%)
    \item Methods (30\%)
    \item Results (35\%)
    \item Discussion (15\%)
    \item Quality of writing (10\%)
\end{itemize}

% \subsection*{References}

\end{document}
